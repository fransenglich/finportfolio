
% TODO A4
\documentclass[A4]{article}

\usepackage[utf8]{inputenc}
\usepackage[english]{babel}
\usepackage[bookmarks=true]{hyperref}
\usepackage{color}
\usepackage{listings}


\usepackage{xcolor}
%New colors defined below
\definecolor{codegreen}{rgb}{0,0.6,0}
\definecolor{codegray}{rgb}{0.5,0.5,0.5}
\definecolor{codepurple}{rgb}{0.58,0,0.82}
\definecolor{backcolour}{rgb}{0.95,0.95,0.92}

%Code listing style named "mystyle"
\lstdefinestyle{mystyle}{
%  backgroundcolor=\color{backcolour},
  commentstyle=\color{codegreen},
  keywordstyle=\color{magenta},
  numberstyle=\tiny\color{codegray},
  stringstyle=\color{codepurple},
  basicstyle=\ttfamily\footnotesize,
  breakatwhitespace=false,
  breaklines=true,
  keepspaces=true,
  numbers=left,
  numbersep=5pt,
  showspaces=false,
  showstringspaces=false,
  showtabs=false,
  tabsize=2
}

%"mystyle" code listing set
\lstset{style=mystyle}

\def\documenttitle{Frans' Savings Plan}

\title{\documenttitle}
\date{\today}
\author{Frans Englich \\
        \href{mailto:fenglich@fastmail.fm}{fenglich@fastmail.fm}}

\hypersetup{
    pdfsubject = {\documenttitle},
    pdftitle = {\documenttitle}
}

\begin{document}

\maketitle

My idea is it would be wise to save for pension or large investment such as an apartment. My interest in this, beyond the end result, is limited: I have no intension to spend time on fundamental analysis or follow developments on a regular basis. However, I do write this document, and I intend to do a modern portfolio theory (MPT) analysis of the portfolio. I have in interest to sleep well, without worrying about this.

Individual stocks are, as far as I can tell, hence irrelevant because I won't manage it, rebalance, and so forth. Therefore I think it makes sense to look at passive indices and active management.

The computations contains no strategic decisions, they are a procedural, technical matter. The interesting questions is what assets to choose, and that is the hard part, not calculating the portfolio weights.

\section{Details}

This is the broad picture:

\begin{itemize}
\item "Long" horizon. Easily 5-10 years
\item Generally offensive/"high" risk investment
\item Bank of choice is Avanza in Sweden
\item Other banks of mine are Nordea in Sweden and Norway. A student's economy, only savings accounts
\item Account type for the investments: Investeringssparkonto (ISK)
\item Start amount: 10 000 SEK
\item Monthly investment: 500 SEK.
\end{itemize}

\section{Investment Objects}

\begin{center}
\begin{tabular}{ |l|l|l| }
    \hline
    \href{https://www.avanza.se/aktier/omc-aktien.html/338588/creades-a}{Creades A} & CRED A & Investment company \\
    \href{https://www.avanza.se/aktier/om-aktien.html/5245/industrivarden-c}{Industrivärden C} & INDU C & Investment company \\
    \href{https://www.avanza.se/fonder/om-fonden.html/929/seb-europafond-smabolag}{SEB Europafond Småbolag} & - & Fund \\
    \href{https://www.avanza.se/fonder/om-fonden.html/70331/ms-invf-us-growth-a-usd}{MS INVF US Growth A USD} & - & Fund \\
    \hline
\end{tabular}
\end{center}

In short, my very shallow approach to this is that it achieves diversification across USA, Europe and Sweden, and is high return/risk and therefore matches my objectives.

\section{A Modern Portfolio Theory Analysis}

The objective is to maximise the portfolio's Sharpe ratio and acquire informational key figures. Therefore I will compute the portfolio's:

\begin{itemize}
    \item Mean/Variance-efficient asset weights
    \item Expected return
    \item Standard deviation.
% VaR?
\end{itemize}

\section{Remaining Questions}

\appendix

\section{Code}

\lstinputlisting[language=Matlab, caption=Matlab code]{frontier.m}


\end{document}
